\documentclass[a4paper,14pt, unknownkeysallowed]{extreport}

\usepackage{cmap} % Улучшенный поиск русских слов в полученном pdf-файле
\usepackage[T2A]{fontenc} % Поддержка русских букв
\usepackage[utf8]{inputenc} % Кодировка utf8
\usepackage[english,russian]{babel} % Языки: русский, английский
\usepackage{enumitem}


\usepackage{threeparttable}

\usepackage[14pt]{extsizes}

\usepackage{caption}
\captionsetup{labelsep=endash}
\captionsetup[figure]{name={Рисунок}}

% \usepackage{ctable}
% \captionsetup[table]{justification=raggedleft,singlelinecheck=off}

\usepackage{amsmath}

\usepackage{geometry}
\geometry{left=30mm}
\geometry{right=10mm}
\geometry{top=20mm}
\geometry{bottom=20mm}

\usepackage{titlesec}
\titleformat{\section}
	{\normalsize\bfseries}
	{\thesection}
	{1em}{}
\titlespacing*{\chapter}{0pt}{-30pt}{8pt}
\titlespacing*{\section}{\parindent}{*4}{*4}
\titlespacing*{\subsection}{\parindent}{*4}{*4}

\usepackage{setspace}
\onehalfspacing % Полуторный интервал

\frenchspacing
\usepackage{indentfirst} % Красная строка

\usepackage{titlesec}
\titleformat{\chapter}{\LARGE\bfseries}{\thechapter}{20pt}{\LARGE\bfseries}
\titleformat{\section}{\Large\bfseries}{\thesection}{20pt}{\Large\bfseries}

\usepackage{multirow}
\usepackage{listings}
\usepackage{xcolor}

% Для листинга кода:
\lstset{%
	language=Lisp,   					% выбор языка для подсветки	
	basicstyle=\small\sffamily,			% размер и начертание шрифта для подсветки кода
	numbers=left,						% где поставить нумерацию строк (слева\справа)
	numberstyle=\tiny,		     		% размер шрифта для номеров строк
	stepnumber=1,						% размер шага между двумя номерами строк
	numbersep=5pt,						% как далеко отстоят номера строк от подсвечиваемого кода
	frame=single,						% рисовать рамку вокруг кода
	tabsize=4,							% размер табуляции по умолчанию равен 4 пробелам
	captionpos=t,						% позиция заголовка вверху [t] или внизу [b]
	breaklines=true,					
	breakatwhitespace=true,				% переносить строки только если есть пробел
	backgroundcolor=\color{white},
	basicstyle=\footnotesize\ttfamily,
	keywordstyle=\color{blue},
	stringstyle=\color{red},
	commentstyle=\color{gray},
	showspaces=false,
    showstringspaces=false
}


\usepackage{pgfplots}
\usetikzlibrary{datavisualization}
\usetikzlibrary{datavisualization.formats.functions}


\lstset{
	literate=
	{а}{{\selectfont\char224}}1
	{б}{{\selectfont\char225}}1
	{в}{{\selectfont\char226}}1
	{г}{{\selectfont\char227}}1
	{д}{{\selectfont\char228}}1
	{е}{{\selectfont\char229}}1
	{ж}{{\selectfont\char230}}1
	{з}{{\selectfont\char231}}1
	{и}{{\selectfont\char232}}1
	{й}{{\selectfont\char233}}1
	{к}{{\selectfont\char234}}1
	{л}{{\selectfont\char235}}1
	{м}{{\selectfont\char236}}1
	{н}{{\selectfont\char237}}1
	{о}{{\selectfont\char238}}1
	{п}{{\selectfont\char239}}1
	{р}{{\selectfont\char240}}1
	{с}{{\selectfont\char241}}1
	{т}{{\selectfont\char242}}1
	{у}{{\selectfont\char243}}1
	{ф}{{\selectfont\char244}}1
	{х}{{\selectfont\char245}}1
	{ц}{{\selectfont\char246}}1
	{ч}{{\selectfont\char247}}1
	{ш}{{\selectfont\char248}}1
	{щ}{{\selectfont\char249}}1
	{ъ}{{\selectfont\char250}}1
	{ы}{{\selectfont\char251}}1
	{ь}{{\selectfont\char252}}1
	{э}{{\selectfont\char253}}1
	{ю}{{\selectfont\char254}}1
	{я}{{\selectfont\char255}}1
	{А}{{\selectfont\char192}}1
	{Б}{{\selectfont\char193}}1
	{В}{{\selectfont\char194}}1
	{Г}{{\selectfont\char195}}1
	{Д}{{\selectfont\char196}}1
	{Е}{{\selectfont\char197}}1
	{Ж}{{\selectfont\char198}}1
	{З}{{\selectfont\char199}}1
	{И}{{\selectfont\char200}}1
	{Й}{{\selectfont\char201}}1
	{К}{{\selectfont\char202}}1
	{Л}{{\selectfont\char203}}1
	{М}{{\selectfont\char204}}1
	{Н}{{\selectfont\char205}}1
	{О}{{\selectfont\char206}}1
	{П}{{\selectfont\char207}}1
	{Р}{{\selectfont\char208}}1
	{С}{{\selectfont\char209}}1
	{Т}{{\selectfont\char210}}1
	{У}{{\selectfont\char211}}1
	{Ф}{{\selectfont\char212}}1
	{Х}{{\selectfont\char213}}1
	{Ц}{{\selectfont\char214}}1
	{Ч}{{\selectfont\char215}}1
	{Ш}{{\selectfont\char216}}1
	{Щ}{{\selectfont\char217}}1
	{Ъ}{{\selectfont\char218}}1
	{Ы}{{\selectfont\char219}}1
	{Ь}{{\selectfont\char220}}1
	{Э}{{\selectfont\char221}}1
	{Ю}{{\selectfont\char222}}1
	{Я}{{\selectfont\char223}}1
}

\usepackage{graphicx}
\newcommand{\img}[3] {
	\begin{figure}[h!]
		\center{\includegraphics[height=#1]{img/#2}}
		\caption{#3}
		\label{img:#2}
	\end{figure}
}


\usepackage[justification=centering]{caption} % Настройка подписей float объектов

\usepackage[unicode,pdftex]{hyperref} % Ссылки в pdf
\hypersetup{hidelinks}

\usepackage{csvsimple}

\newcommand{\code}[1]{\texttt{#1}}

\usepackage{longtable}

\usepackage{array}
\usepackage{booktabs}
\usepackage{floatrow}

\floatsetup[longtable]{LTcapwidth=table}

% \def\UrlBreaks{\do\/\do-\do\_}

\makeatletter
\renewcommand*\l@chapter[2]{%
  \ifnum \c@tocdepth >\m@ne
    \addpenalty{-\@highpenalty}%
    \vskip 1.0em \@plus\p@
    \setlength\@tempdima{1.5em}%
    \begingroup
      \parindent \z@ \rightskip \@pnumwidth
      \parfillskip -\@pnumwidth
      \leavevmode \bfseries
      \advance\leftskip\@tempdima
      \hskip -\leftskip
      #1\nobreak\normalfont\leaders\hbox{$\m@th
        \mkern \@dotsep mu\hbox{.}\mkern \@dotsep
        mu$}\hfill\nobreak\hb@xt@\@pnumwidth{\hss #2}\par
      \penalty\@highpenalty
    \endgroup
  \fi}
\makeatother

\begin{document}



\begin{titlepage}
	\newgeometry{pdftex, left=2cm, right=2cm, top=2.5cm, bottom=2.5cm}
	\fontsize{12pt}{12pt}\selectfont
	\noindent \begin{minipage}{0.15\textwidth}
		\includegraphics[width=\linewidth]{img/b_logo.jpg}
	\end{minipage}
	\noindent\begin{minipage}{0.9\textwidth}\centering
		\textbf{Министерство науки и высшего образования Российской Федерации}\\
		\textbf{Федеральное государственное бюджетное образовательное учреждение высшего образования}\\
		\textbf{«Московский государственный технический университет имени Н. Э.~Баумана}\\
		\textbf{(национальный исследовательский университет)»}\\
		\textbf{(МГТУ им. Н. Э.~Баумана)}
	\end{minipage}
	
	\noindent\rule{18cm}{3pt}
	\newline\newline
	\noindent ФАКУЛЬТЕТ $\underline{\text{«Информатика и системы управления»~~~~~~~~~~~~~~~~~~~~~~~~~~~~~~~~~~~~~~~~~~~~~~~~~~~~~~~}}$ \newline\newline
	\noindent КАФЕДРА $\underline{\text{«Программное обеспечение ЭВМ и информационные технологии»~~~~~~~~~~~~~~~~~~~~~~~}}$\newline\newline\newline\newline\newline\newline\newline
	
	
	\begin{center}
		\noindent\begin{minipage}{1.3\textwidth}\centering
		\Large\textbf{   ~~~ Лабораторная работа №3}\newline
		\textbf{по курсу "Функциональное}\newline
		\textbf{и логическое программирование"}\newline\newline\newline
		\end{minipage}
	\end{center}
	
	\noindent\textbf{Тема} 			$\underline{\text{Работа интерпретатора Lisp}}$\newline\newline
	\noindent\textbf{Студент} 		$\underline{\text{Ковалец К. Э.}}$\newline\newline
	\noindent\textbf{Группа} 		$\underline{\text{ИУ7-63Б}}$\newline\newline
	\noindent\textbf{Преподаватели} $\underline{\text{Толпинская Н. Б., Строганов Ю. В.}}$\newline
	
	\begin{center}
		\vfill
		Москва~---~\the\year
		~г.
	\end{center}
	\restoregeometry
\end{titlepage}



\setcounter{page}{2}
\chapter{Практические задания}

\section{Задание 1}

Написать функцию, которая принимает целое число и возвращает первое четное число, не меньшее аргумента.

\begin{center}
\captionsetup{justification=raggedright,singlelinecheck=off}
\begin{lstlisting}[label=lst:parallel_processing,caption=Решение задания 1]
(defun first-even (num)
    (if (evenp num) num (+ num 1)))

;; (FIRST-EVEN 1) -> 2
;; (FIRST-EVEN 2) -> 2
\end{lstlisting}
\end{center}

\section{Задание 2}

Написать функцию, которая принимает число и возвращает число того же знака, но с модулем на 1 больше модуля аргумента.

\begin{center}
\captionsetup{justification=raggedright,singlelinecheck=off}
\begin{lstlisting}[label=lst:parallel_processing,caption=Решение задания 2]
(defun module-plus (num)
    (+ num (if (> num 0) 1 -1)))

;; (MODULE-PLUS 1) -> 2
;; (MODULE-PLUS -1) -> -2
\end{lstlisting}
\end{center}

\section{Задание 3}

Написать функцию, которая принимает два числа и возвращает список из этих чисел, расположенный по возрастанию.

\begin{center}
\captionsetup{justification=raggedright,singlelinecheck=off}
\begin{lstlisting}[label=lst:parallel_processing,caption=Решение задания 3]
(defun ascending-list (num1 num2)
    (if (> num1 num2) 
		(list num2 num1) (list num1 num2)))

;; (ASCENDING-LIST 1 2) -> (1 2)
;; (ASCENDING-LIST 2 1) -> (1 2)
\end{lstlisting}
\end{center}

\section{Задание 4}

Написать функцию, которая принимает три числа и возвращает Т только тогда, когда первое число расположено между вторым и третьим.

\begin{center}
\captionsetup{justification=raggedright,singlelinecheck=off}
\begin{lstlisting}[label=lst:parallel_processing,caption=Решение задания 4]
(defun between (num1 num2 num3)
    (or (and (> num1 num2) (< num1 num3))
        (and (< num1 num2) (> num1 num3))))

;; (BETWEEN 2 1 3) -> T
;; (BETWEEN 2 3 1) -> T
;; (BETWEEN 1 2 3) -> NIL
;; (BETWEEN 4 3 2) -> NIL
\end{lstlisting}
\end{center}

\section{Задание 5}

Каков результат вычисления следующих выражений?

\begin{center}
\captionsetup{justification=raggedright,singlelinecheck=off}
\begin{lstlisting}[label=lst:parallel_processing,caption=Решение задания 5]
(and 'fee 'fie 'foe)         ;; FOE
(or nil 'fie 'foe)           ;; FIE
(and (equal 'abc 'abc) 'yes) ;; YES
(or 'fee 'fie 'foe)          ;; FEE
(and nil 'fie 'foe)          ;; NIL
(or (equal 'abc 'abc) 'yes)  ;; T
\end{lstlisting}
\end{center}

\section{Задание 6}

Написать предикат, который принимает два числа-аргумента и возвращает Т, если первое число не меньше второго.

\begin{center}
\captionsetup{justification=raggedright,singlelinecheck=off}
\begin{lstlisting}[label=lst:parallel_processing,caption=Решение задания 6]
(defun predicate (num1 num2)
    (>= num1 num2))

;; (PREDICATE 1 2) -> NIL
;; (PREDICATE 2 1) -> T
\end{lstlisting}
\end{center}

\section{Задание 7}

Какой из следующих двух вариантов предиката ошибочен и почему?

\begin{itemize}
	\item \textbf{numberp} — возвращает \texttt{T}, если значение аргумента — числовой атом, \texttt{Nil} иначе;
	\item \textbf{plusp} — возвращает \texttt{T}, если аргумент больше нуля, \texttt{Nil} иначе.
\end{itemize}

\begin{center}
\captionsetup{justification=raggedright,singlelinecheck=off}
\begin{lstlisting}[label=lst:parallel_processing,caption=Решение задания 7]
(defun pred1 (x) 
    (and (numberp x) (plusp x))) ;; OK

(defun pred2 (x)
    (and (plusp x) (numberp x)))  ;; ERROR

;; (PRED1 1) -> T
;; (PRED2 1) -> T

;; (PRED1 'Hello) -> NIL
;; (PRED2 'Hello) -> The value HELLO is not of type NUMBER
\end{lstlisting}
\end{center}

Во втором случае при попытке проверить, является ли аргумент больше нуля, может возникнуть ошибка, если аргумент не является числовым атомом, так как при вызове функции \texttt{and} аргументы вычисляются слева направо до первого вернувшего \texttt{Nil}


\section{Задание 8}

Решить задачу 4, используя для ее решения конструкции IF, COND, AND/OR. Написать функцию, которая принимает три числа и возвращает Т только тогда, когда первое число расположено между вторым и третьим.

\begin{center}
\captionsetup{justification=raggedright,singlelinecheck=off}
\begin{lstlisting}[label=lst:parallel_processing,caption=Решение задания 8]
;; С использованием IF

(defun between (num1 num2 num3)
    (if (> num1 num2) 
        (< num1 num3)

        (if (< num1 num2) 
            (> num1 num3))))

;; С использованием COND

(defun between (num1 num2 num3)
	(cond ((> num1 num2) (< num1 num3))
		  ((< num1 num2) (> num1 num3))))

;; С использованием AND/OR

(defun between (num1 num2 num3)
	(or (and (> num1 num2) (< num1 num3))
		(and (< num1 num2) (> num1 num3))))

;; (BETWEEN 2 1 3) -> T
;; (BETWEEN 2 3 1) -> T
;; (BETWEEN 1 2 3) -> NIL
;; (BETWEEN 4 3 2) -> NIL
\end{lstlisting}
\end{center}

\clearpage

\section{Задание 9}

Переписать функцию how-alike, приведенную в лекции и использующую COND, используя только конструкции IF, AND/OR.

\begin{center}
\captionsetup{justification=raggedright,singlelinecheck=off}
\begin{lstlisting}[label=lst:parallel_processing,caption=Функция how-alike из лекции]
(defun how_alike(x y)
    (cond ((or  (= x y) (equal x y)) 'the_same)
          ((and (oddp x) (oddp y))   'both_odd) 
          ((and (evenp x) (evenp y)) 'both_even) 
          (t 'difference)))
\end{lstlisting}
\end{center}


\begin{center}
\captionsetup{justification=raggedright,singlelinecheck=off}
\begin{lstlisting}[label=lst:parallel_processing,caption=Решение задания 9]
;; С использованием IF

(defun how_alike(x y)
	(if (or  (= x y) (equal x y)) 'the_same 
	(if (and (oddp x) (oddp y))   'both_odd 
	(if (and (evenp x) (evenp y)) 'both_even 'difference))))

;; С использованием AND/OR

(defun how_alike(x y)
	(or (and (or  (= x y) (equal x y)) 'the_same)
		(and (and (oddp x) (oddp y))   'both_odd)
		(and (and (evenp x) (evenp y)) 'both_even)
		'difference))

;; (HOW_ALIKE 4 4) -> THE_SAME
;; (HOW_ALIKE 4 6) -> BOTH_EVEN
;; (HOW_ALIKE 1 3) -> BOTH_ODD
;; (HOW_ALIKE 4 5) -> DIFFERENCE	
\end{lstlisting}
\end{center}


\chapter{Ответы на теоретические вопросы к лабораторной работе}

\section{Базис Lisp}

\textbf{Базис языка} -- минимальный набор конструкций языка и структур данных, с помощью которых можно решить любую задачу.

Базис языка Lisp содержит:

\begin{itemize}
	\item атомы и структуры (представляющиеся бинарными узлами);
	\item базовые функции и функционалы:
	\begin{itemize}
		\item встроенные -- примитивные функции (atom, eq, cons, car, cdr);
		\item специальные функции и функционалы (quote, cond, lambda, eval, apply, funcall).
	\end{itemize}
\end{itemize}

\section{Классификация функций}

Функции в Lisp классифицируют следующим образом:

\begin{itemize}
	\item чистые математические функции (имеют фиксированное кол-во аргументов и один результат);
	\item рекурсивные функции;
	\item специальные функции -- формы (прнимают произвольное число аргументов или по разному обрабатывают аргументы);
	\item псевдофункции (создают эффект на внешнем устройстве);
	\item функции с вариативными значениями, из которых выбирается одно;
	\item функции высших порядков -- функционалы (используются для создания синтаксически управляемых программ).
\end{itemize}

По назначению функции разделяются следующим образом:

\begin{itemize}
	\item конструкторы — создают значение (cons, list);
	\item селекторы — получают доступ по адресу (car, cdr); 
	\item предикаты — возвращают Nil, T.
\end{itemize}

\section{Способы создания функций}

\textbf{Функцией} называется правило, по которому каждому значению одного или нескольких аргументов ставится в соответствие конкретное значение результата.

\begin{itemize}
	\item В Lisp можно определить функцию без имени с помощью \textbf{$\lambda$-выражений}. 
	Lambda-определение безымянной функции:
	
	\begin{center}
	\texttt{(lambda <lambda-список> <форма>)}
	\end{center}

	Lambda-вызов функции:

	\begin{center}
	\texttt{(<lambda-выражение> <формальные параметры>)}
	\end{center}

	\item Также в Lisp можно определить функцию с именем с помощью \textbf{defun}. В таких функциях defun связывает символьный атом с Lambda-определением:
	
	\begin{center}
	\texttt{(defun f <lambda-выражение>)}
	\end{center}

	Упрощенное определение:

	\begin{center}
	\texttt{(defun f(arg1, ..., argN) <формы>)}
	\end{center}

\end{itemize}

\section{Работа функций Cond, if, and/or}

\begin{itemize}
	\item \textbf{cond}
	

	\begin{center}
	\captionsetup{justification=raggedright,singlelinecheck=off}
	\begin{lstlisting}
(cond (test1 body1) 
	  (test2 body2)
	   ...
	  (testN bodyN)
	 [(T else-body)])
	\end{lstlisting}
	\end{center}

	Список аргументов обрабатывается последовательно: вычисляется выражение \texttt{test\_i}, и если не \texttt{Nil}, то вычисляется \texttt{body\_i}, и работа функции завершается, если ни один тест не выполнился, то возвращается \texttt{Nil}, можно организовать ветку <<\texttt{else}>>, явно указав в качестве \texttt{test -- Т}.

	\item \textbf{if}
	
	\begin{center}
	\captionsetup{justification=raggedright,singlelinecheck=off}
	\begin{lstlisting}
(if test T-body F-body)
	\end{lstlisting}
	\end{center}

	Работа функции \texttt{if} очевидна, с учетом, что всё что не nil, то T. Результат теста может быть как атомом (не обязательно Nil) так и списком. В зависимости от test, будет вычислен либо один либо другой аргумент.

	\item \textbf{and}
	
	\begin{center}
	\captionsetup{justification=raggedright,singlelinecheck=off}
	\begin{lstlisting}
(and arg1 arg2 ... argN) 
	\end{lstlisting}
	\end{center}

	Функция \texttt{and} вычисляет аргументы, пока не станет очевидным результат (появится первый \texttt{nil}). Как только станет очевиден результат -- возвращается последнее вычисленное значение.

	\item \textbf{or}
	
	\begin{center}
	\captionsetup{justification=raggedright,singlelinecheck=off}
	\begin{lstlisting}
(or arg1 arg2 ... argN)
	\end{lstlisting}
	\end{center}

	Функция \texttt{or} вычисляет аргументы, пока не станет очевидным результат (появится первый не \texttt{nil}). Как только станет очевиден результат -- возвращается последнее вычисленное значение.

\end{itemize}


\end{document}
