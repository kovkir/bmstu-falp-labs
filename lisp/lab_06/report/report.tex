\documentclass[a4paper,14pt, unknownkeysallowed]{extreport}

\usepackage{cmap} % Улучшенный поиск русских слов в полученном pdf-файле
\usepackage[T2A]{fontenc} % Поддержка русских букв
\usepackage[utf8]{inputenc} % Кодировка utf8
\usepackage[english,russian]{babel} % Языки: русский, английский
\usepackage{enumitem}


\usepackage{threeparttable}

\usepackage[14pt]{extsizes}

\usepackage{caption}
\captionsetup{labelsep=endash}
\captionsetup[figure]{name={Рисунок}}

% \usepackage{ctable}
% \captionsetup[table]{justification=raggedleft,singlelinecheck=off}

\usepackage{amsmath}

\usepackage{geometry}
\geometry{left=30mm}
\geometry{right=10mm}
\geometry{top=20mm}
\geometry{bottom=20mm}

\usepackage{titlesec}
\titleformat{\section}
	{\normalsize\bfseries}
	{\thesection}
	{1em}{}
\titlespacing*{\chapter}{0pt}{-30pt}{8pt}
\titlespacing*{\section}{\parindent}{*4}{*4}
\titlespacing*{\subsection}{\parindent}{*4}{*4}

\usepackage{setspace}
\onehalfspacing % Полуторный интервал

\frenchspacing
\usepackage{indentfirst} % Красная строка

\usepackage{titlesec}
\titleformat{\chapter}{\LARGE\bfseries}{\thechapter}{20pt}{\LARGE\bfseries}
\titleformat{\section}{\Large\bfseries}{\thesection}{20pt}{\Large\bfseries}

\usepackage{multirow}
\usepackage{listings}
\usepackage{xcolor}

% Для листинга кода:
\lstset{%
	language=Lisp,   					% выбор языка для подсветки	
	basicstyle=\small\sffamily,			% размер и начертание шрифта для подсветки кода
	numbers=left,						% где поставить нумерацию строк (слева\справа)
	numberstyle=\tiny,		     		% размер шрифта для номеров строк
	stepnumber=1,						% размер шага между двумя номерами строк
	numbersep=5pt,						% как далеко отстоят номера строк от подсвечиваемого кода
	frame=single,						% рисовать рамку вокруг кода
	tabsize=4,							% размер табуляции по умолчанию равен 4 пробелам
	captionpos=t,						% позиция заголовка вверху [t] или внизу [b]
	breaklines=true,					
	breakatwhitespace=true,				% переносить строки только если есть пробел
	backgroundcolor=\color{white},
	basicstyle=\footnotesize\ttfamily,
	keywordstyle=\color{blue},
	stringstyle=\color{red},
	commentstyle=\color{gray},
	showspaces=false,
    showstringspaces=false
}


\usepackage{pgfplots}
\usetikzlibrary{datavisualization}
\usetikzlibrary{datavisualization.formats.functions}


\lstset{
	literate=
	{а}{{\selectfont\char224}}1
	{б}{{\selectfont\char225}}1
	{в}{{\selectfont\char226}}1
	{г}{{\selectfont\char227}}1
	{д}{{\selectfont\char228}}1
	{е}{{\selectfont\char229}}1
	{ж}{{\selectfont\char230}}1
	{з}{{\selectfont\char231}}1
	{и}{{\selectfont\char232}}1
	{й}{{\selectfont\char233}}1
	{к}{{\selectfont\char234}}1
	{л}{{\selectfont\char235}}1
	{м}{{\selectfont\char236}}1
	{н}{{\selectfont\char237}}1
	{о}{{\selectfont\char238}}1
	{п}{{\selectfont\char239}}1
	{р}{{\selectfont\char240}}1
	{с}{{\selectfont\char241}}1
	{т}{{\selectfont\char242}}1
	{у}{{\selectfont\char243}}1
	{ф}{{\selectfont\char244}}1
	{х}{{\selectfont\char245}}1
	{ц}{{\selectfont\char246}}1
	{ч}{{\selectfont\char247}}1
	{ш}{{\selectfont\char248}}1
	{щ}{{\selectfont\char249}}1
	{ъ}{{\selectfont\char250}}1
	{ы}{{\selectfont\char251}}1
	{ь}{{\selectfont\char252}}1
	{э}{{\selectfont\char253}}1
	{ю}{{\selectfont\char254}}1
	{я}{{\selectfont\char255}}1
	{А}{{\selectfont\char192}}1
	{Б}{{\selectfont\char193}}1
	{В}{{\selectfont\char194}}1
	{Г}{{\selectfont\char195}}1
	{Д}{{\selectfont\char196}}1
	{Е}{{\selectfont\char197}}1
	{Ж}{{\selectfont\char198}}1
	{З}{{\selectfont\char199}}1
	{И}{{\selectfont\char200}}1
	{Й}{{\selectfont\char201}}1
	{К}{{\selectfont\char202}}1
	{Л}{{\selectfont\char203}}1
	{М}{{\selectfont\char204}}1
	{Н}{{\selectfont\char205}}1
	{О}{{\selectfont\char206}}1
	{П}{{\selectfont\char207}}1
	{Р}{{\selectfont\char208}}1
	{С}{{\selectfont\char209}}1
	{Т}{{\selectfont\char210}}1
	{У}{{\selectfont\char211}}1
	{Ф}{{\selectfont\char212}}1
	{Х}{{\selectfont\char213}}1
	{Ц}{{\selectfont\char214}}1
	{Ч}{{\selectfont\char215}}1
	{Ш}{{\selectfont\char216}}1
	{Щ}{{\selectfont\char217}}1
	{Ъ}{{\selectfont\char218}}1
	{Ы}{{\selectfont\char219}}1
	{Ь}{{\selectfont\char220}}1
	{Э}{{\selectfont\char221}}1
	{Ю}{{\selectfont\char222}}1
	{Я}{{\selectfont\char223}}1
}

\usepackage{graphicx}
\newcommand{\img}[3] {
	\begin{figure}[h!]
		\center{\includegraphics[height=#1]{img/#2}}
		\caption{#3}
		\label{img:#2}
	\end{figure}
}


\usepackage[justification=centering]{caption} % Настройка подписей float объектов

\usepackage[unicode,pdftex]{hyperref} % Ссылки в pdf
\hypersetup{hidelinks}

\usepackage{csvsimple}

\newcommand{\code}[1]{\texttt{#1}}

\usepackage{longtable}

\usepackage{array}
\usepackage{booktabs}
\usepackage{floatrow}

\floatsetup[longtable]{LTcapwidth=table}

% \def\UrlBreaks{\do\/\do-\do\_}

\makeatletter
\renewcommand*\l@chapter[2]{%
  \ifnum \c@tocdepth >\m@ne
    \addpenalty{-\@highpenalty}%
    \vskip 1.0em \@plus\p@
    \setlength\@tempdima{1.5em}%
    \begingroup
      \parindent \z@ \rightskip \@pnumwidth
      \parfillskip -\@pnumwidth
      \leavevmode \bfseries
      \advance\leftskip\@tempdima
      \hskip -\leftskip
      #1\nobreak\normalfont\leaders\hbox{$\m@th
        \mkern \@dotsep mu\hbox{.}\mkern \@dotsep
        mu$}\hfill\nobreak\hb@xt@\@pnumwidth{\hss #2}\par
      \penalty\@highpenalty
    \endgroup
  \fi}
\makeatother

\begin{document}



\begin{titlepage}
	\newgeometry{pdftex, left=2cm, right=2cm, top=2.5cm, bottom=2.5cm}
	\fontsize{12pt}{12pt}\selectfont
	\noindent \begin{minipage}{0.15\textwidth}
		\includegraphics[width=\linewidth]{img/b_logo.jpg}
	\end{minipage}
	\noindent\begin{minipage}{0.9\textwidth}\centering
		\textbf{Министерство науки и высшего образования Российской Федерации}\\
		\textbf{Федеральное государственное бюджетное образовательное учреждение высшего образования}\\
		\textbf{«Московский государственный технический университет имени Н. Э.~Баумана}\\
		\textbf{(национальный исследовательский университет)»}\\
		\textbf{(МГТУ им. Н. Э.~Баумана)}
	\end{minipage}
	
	\noindent\rule{18cm}{3pt}
	\newline\newline
	\noindent ФАКУЛЬТЕТ $\underline{\text{«Информатика и системы управления»~~~~~~~~~~~~~~~~~~~~~~~~~~~~~~~~~~~~~~~~~~~~~~~~~~~~~~~}}$ \newline\newline
	\noindent КАФЕДРА $\underline{\text{«Программное обеспечение ЭВМ и информационные технологии»~~~~~~~~~~~~~~~~~~~~~~~}}$\newline\newline\newline\newline\newline\newline\newline
	
	
	\begin{center}
		\noindent\begin{minipage}{1.3\textwidth}\centering
		\Large\textbf{   ~~~ Лабораторная работа №6}\newline
		\textbf{по курсу "Функциональное}\newline
		\textbf{и логическое программирование"}\newline\newline\newline
		\end{minipage}
	\end{center}
	
	\noindent\textbf{Тема} 			$\underline{\text{Использование функционалов}}$\newline\newline
	\noindent\textbf{Студент} 		$\underline{\text{Ковалец К. Э.}}$\newline\newline
	\noindent\textbf{Группа} 		$\underline{\text{ИУ7-63Б}}$\newline\newline
	\noindent\textbf{Преподаватели} $\underline{\text{Толпинская Н. Б., Строганов Ю. В.}}$\newline
	
	\begin{center}
		\vfill
		Москва~---~\the\year
		~г.
	\end{center}
	\restoregeometry
\end{titlepage}



\setcounter{page}{2}

\chapter{Практические задания}


\section{Мои функции}

\begin{center}
\captionsetup{justification=raggedright,singlelinecheck=off}
\begin{lstlisting}[label=lst:parallel_processing,caption=my-reverse]
(defun my-reverse-rec (lst rev-lst)
	(cond 
		((null lst)
			rev-lst)
		(T 
			(my-reverse-rec (cdr lst) (cons (car lst) rev-lst)))
	)
)

(defun my-reverse (lst)
	(my-reverse-rec lst nil))

;; (MY-REVERSE '(1 2 3 4 5)) -> (5 4 3 2 1)
\end{lstlisting}
\end{center}


\begin{center}
\captionsetup{justification=raggedright,singlelinecheck=off}
\begin{lstlisting}[label=lst:parallel_processing,caption=my-append]
(defun my-append (lst1 lst2)
	(cond 
		((null lst1)
			lst2)
		(T 
			(cons (car lst1) (my-append (cdr lst1) lst2)))
	)
)

;; (MY-APPEND '(1 2 3) '(4 5 6)) -> (1 2 3 4 5 6)
\end{lstlisting}
\end{center}

\clearpage

\begin{center}
\captionsetup{justification=raggedright,singlelinecheck=off}
\begin{lstlisting}[label=lst:parallel_processing,caption=my-sort]
(defun find-min-elem-rec (lst cur_min)
	(cond 
		((null lst)
			cur_min)
		(T 
			(find-min-elem-rec (cdr lst) 
				(cond 
					((< (car lst) cur_min)
						(setf cur_min (car lst)))
					(T 
						cur_min))))
	)
)

(defun find-min-elem (lst)
	(find-min-elem-rec lst (car lst)))

(defun insert (lst elem elem_instead)
	(cond 
		((null lst)
			elem_instead)
		((eql (car lst) elem_instead)
			(setf (car lst) elem))
		(T 
			(insert (cdr lst) elem elem_instead))
	)
)

(defun my-sort-rec (res_lst lst)
	(let* ((min_elem (find-min-elem lst)))
		(cond 
			((null lst)
				res_lst)
			(T 
				(insert lst (car lst) min_elem)
				(my-sort-rec 
					(my-append res_lst (list min_elem))
					(cdr lst)))
		)
	)
)

(defun my-sort (lst)
    (my-sort-rec NIL lst))

;; (MY-SORT '(4 7 2 2 8 1 3)) -> (1 2 2 3 4 7 8)	
\end{lstlisting}
\end{center}

Используя функционалы:

\section{Задание 1}

Напишите функцию, которая уменьшает на 10 все числа из списка-аргумента этой функции.

\begin{center}
\captionsetup{justification=raggedright,singlelinecheck=off}
\begin{lstlisting}[label=lst:parallel_processing,caption=Решение задания 1]
(defun reduce-by-ten (lst)
    (mapcar #'(lambda (x) 
        (cond 
            ((numberp x)
                (- x 10))
            ((listp x)
                (reduce-by-ten x))
            (T 
                x))) lst))

;; (REDUCE-BY-TEN '(10 ((11) 12))) -> (0 1 2)
\end{lstlisting}
\end{center}

\section{Задание 2}

Напишите функцию, которая умножает на заданное число-аргумент все числа из заданного списка-аргумента, когда

\begin{itemize}
	\item все элементы списка -- числа,
	\item элементы списка -- любые объекты.
\end{itemize}

\begin{center}
\captionsetup{justification=raggedright,singlelinecheck=off}
\begin{lstlisting}[label=lst:parallel_processing,caption=Решение задания 2]
;; a) все элементы списка -- числа

(defun multiply-by (lst numb)
	(mapcar #'(lambda (x) (* x numb)) lst))

;; (MULTIPLY-BY '(1 2 3) 10) -> (10 20 30)


;; б) элементы списка -- любые объекты

(defun multiply-by (lst numb)
    (mapcar 
        #'(lambda (x) 
            (cond 
                ((numberp x)
                    (* x numb))
                ((listp x)
                    (multiply-by x numb))
                (T x)
            )
        ) lst
    )
)

;; (MULTIPLY-BY '(((1 2) 3) (4 5) a) 10) -> (((10 20) 30) (40 50) A)
\end{lstlisting}
\end{center}

\section{Задание 3}

Написать функцию, которая по своему списку-аргументу lst определяет является ли он палиндромом (то есть равны ли lst и (reverse lst)).

\begin{center}
\captionsetup{justification=raggedright,singlelinecheck=off}
\begin{lstlisting}[label=lst:parallel_processing,caption=Решение задания 3]
(defun compare (lst1 lst2)
    (reduce #'(lambda (b1 b2) (and b1 b2)) 
        (mapcar #'eql lst1 lst2)))

(defun palindrome (lst)
    (apply #'(lambda (lst1 lst2) (compare lst1 lst2)) 
        (list lst (my-reverse lst))))

;; (PALINDROME '(1 2 3)) -> NIL
;; (PALINDROME '(1 2 2 1)) -> T
;; (PALINDROME '(1 2 3 2 1)) -> T
\end{lstlisting}
\end{center}

\clearpage

\section{Задание 4}

Написать предикат set-equal, который возвращает t, если два его множества-аргумента содержат одни и те же элементы, порядок которых не имеет значения.

\begin{center}
\captionsetup{justification=raggedright,singlelinecheck=off}
\begin{lstlisting}[label=lst:parallel_processing,caption=Решение задания 4]
(defun and-lst (lst)
    (reduce #'(lambda (b1 b2) (and b1 b2)) lst))

(defun or-lst (lst)
    (reduce #'(lambda (b1 b2) (or b1 b2)) lst))

(defun subset (set1 set2)
    (and-lst (mapcar #'(lambda (elem1) 
        (or-lst (mapcar #'(lambda (elem2) 
            (eql elem1 elem2)) set2))) set1)))

(defun set-equal (set1 set2)
    (and (subset set1 set2)
         (subset set2 set1)))

;;(SET-EQUAL '(1 2 3 4 5) '(4 2 5 1 3)) -> T
;;(SET-EQUAL '(1 2 3 4 5) '(4 2 5 1 7)) -> NIL
\end{lstlisting}
\end{center}

\section{Задание 5}

Написать функцию которая получает как аргумент список чисел, а возвращает список квадратов этих чисел в том же порядке.

\begin{center}
\captionsetup{justification=raggedright,singlelinecheck=off}
\begin{lstlisting}[label=lst:parallel_processing,caption=Решение задания 5]
(defun square (lst)
    (mapcar #'(lambda (x) (* x x)) lst))

;; (SQUARE '(1 2 3 4)) -> (1 4 9 16)
\end{lstlisting}
\end{center}

\section{Задание 6}

Напишите функцию, select-between, которая из списка-аргумента, содержащего только числа, выбирает только те, которые расположены между двумя указанными границами-аргументами и возвращает их в виде списка (упорядоченного по возрастанию списка чисел (+ 2 балла)).

\begin{center}
\captionsetup{justification=raggedright,singlelinecheck=off}
\begin{lstlisting}[label=lst:parallel_processing,caption=Решение задания 6]
(defun select-between (lst bord1 bord2)
    (my-sort (remove-if #'(lambda (x) 
        (not (or (> bord1 x bord2)
                 (< bord1 x bord2)))) lst)))

;; (SELECT-BETWEEN '(1 5 4 2 3) 2 5) -> (3 4)
;; (SELECT-BETWEEN '(1 5 4 2 3) 0 6) -> (1 2 3 4 5)
;; (SELECT-BETWEEN '(1 5 4 2 3) 7 9) -> NIL
\end{lstlisting}
\end{center}

\section{Задание 7}

Написать функцию, вычисляющую декартово произведение двух своих списков-аргументов. (Напомним, что А х В это множество всевозможных пар (a b), где а принадлежит А, принадлежит В.)

\begin{center}
\captionsetup{justification=raggedright,singlelinecheck=off}
\begin{lstlisting}[label=lst:parallel_processing,caption=Решение задания 7]
(defun decart_mult (lst1 lst2)
    (mapcan #'(lambda (x1) 
        (mapcar #'(lambda (x2) (list x1 x2)) 
            lst2)) lst1))

;; (DECART_MULT '(1 2 3) '(1 2 3)) ->
;; ((1 1) (1 2) (1 3) (2 1) (2 2) (2 3) (3 1) (3 2) (3 3))
\end{lstlisting}
\end{center}

\clearpage

\section{Задание 8}

Почему так реализовано reduce, в чем причина? 

\begin{center}
\captionsetup{justification=raggedright,singlelinecheck=off}
\begin{lstlisting}[label=lst:parallel_processing,caption=Решение задания 8]
(reduce #'+ 0)  ;; -> Error
(reduce #'+ ()) ;; -> 0
\end{lstlisting}
\end{center}

\section{Задание 9}

Пусть list-of-list список, состоящий из списков. Написать функцию, которая вычисляет сумму длин всех элементов list-of-list, т.е. например для аргумента ((1 2) (3 4)) -> 4.

\begin{center}
\captionsetup{justification=raggedright,singlelinecheck=off}
\begin{lstlisting}[label=lst:parallel_processing,caption=Решение задания 9]
(defun sum-len-list-of-list (lst)
    (reduce #'+ (mapcar 
        #'(lambda (x)
            (cond ((listp x) (sum-len-list-of-list x))
                  (T 1))) 
        lst)))    

;; (SUM-LEN-LIST-OF-LIST '((1 2) (3 4))) -> 4
\end{lstlisting}
\end{center}

\end{document}
