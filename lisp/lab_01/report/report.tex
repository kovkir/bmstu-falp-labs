\documentclass[a4paper,14pt, unknownkeysallowed]{extreport}

\usepackage{cmap} % Улучшенный поиск русских слов в полученном pdf-файле
\usepackage[T2A]{fontenc} % Поддержка русских букв
\usepackage[utf8]{inputenc} % Кодировка utf8
\usepackage[english,russian]{babel} % Языки: русский, английский
\usepackage{enumitem}


\usepackage{threeparttable}

\usepackage[14pt]{extsizes}

\usepackage{caption}
\captionsetup{labelsep=endash}
\captionsetup[figure]{name={Рисунок}}

% \usepackage{ctable}
% \captionsetup[table]{justification=raggedleft,singlelinecheck=off}

\usepackage{amsmath}

\usepackage{geometry}
\geometry{left=30mm}
\geometry{right=10mm}
\geometry{top=20mm}
\geometry{bottom=20mm}

\usepackage{titlesec}
\titleformat{\section}
	{\normalsize\bfseries}
	{\thesection}
	{1em}{}
\titlespacing*{\chapter}{0pt}{-30pt}{8pt}
\titlespacing*{\section}{\parindent}{*4}{*4}
\titlespacing*{\subsection}{\parindent}{*4}{*4}

\usepackage{setspace}
\onehalfspacing % Полуторный интервал

\frenchspacing
\usepackage{indentfirst} % Красная строка

\usepackage{titlesec}
\titleformat{\chapter}{\LARGE\bfseries}{\thechapter}{20pt}{\LARGE\bfseries}
\titleformat{\section}{\Large\bfseries}{\thesection}{20pt}{\Large\bfseries}

\usepackage{multirow}
\usepackage{listings}
\usepackage{xcolor}

% Для листинга кода:
\lstset{%
	language=python,   					% выбор языка для подсветки	
	basicstyle=\small\sffamily,			% размер и начертание шрифта для подсветки кода
	numbers=left,						% где поставить нумерацию строк (слева\справа)
	numberstyle=\tiny,		     		% размер шрифта для номеров строк
	stepnumber=1,						% размер шага между двумя номерами строк
	numbersep=5pt,						% как далеко отстоят номера строк от подсвечиваемого кода
	frame=single,						% рисовать рамку вокруг кода
	tabsize=4,							% размер табуляции по умолчанию равен 4 пробелам
	captionpos=t,						% позиция заголовка вверху [t] или внизу [b]
	breaklines=true,					
	breakatwhitespace=true,				% переносить строки только если есть пробел
	backgroundcolor=\color{white},
	basicstyle=\footnotesize\ttfamily,
	keywordstyle=\color{blue},
	stringstyle=\color{red},
	commentstyle=\color{gray}
	showspaces=false,
    showstringspaces=false
}


\usepackage{pgfplots}
\usetikzlibrary{datavisualization}
\usetikzlibrary{datavisualization.formats.functions}


\lstset{
	literate=
	{а}{{\selectfont\char224}}1
	{б}{{\selectfont\char225}}1
	{в}{{\selectfont\char226}}1
	{г}{{\selectfont\char227}}1
	{д}{{\selectfont\char228}}1
	{е}{{\selectfont\char229}}1
	{ж}{{\selectfont\char230}}1
	{з}{{\selectfont\char231}}1
	{и}{{\selectfont\char232}}1
	{й}{{\selectfont\char233}}1
	{к}{{\selectfont\char234}}1
	{л}{{\selectfont\char235}}1
	{м}{{\selectfont\char236}}1
	{н}{{\selectfont\char237}}1
	{о}{{\selectfont\char238}}1
	{п}{{\selectfont\char239}}1
	{р}{{\selectfont\char240}}1
	{с}{{\selectfont\char241}}1
	{т}{{\selectfont\char242}}1
	{у}{{\selectfont\char243}}1
	{ф}{{\selectfont\char244}}1
	{х}{{\selectfont\char245}}1
	{ц}{{\selectfont\char246}}1
	{ч}{{\selectfont\char247}}1
	{ш}{{\selectfont\char248}}1
	{щ}{{\selectfont\char249}}1
	{ъ}{{\selectfont\char250}}1
	{ы}{{\selectfont\char251}}1
	{ь}{{\selectfont\char252}}1
	{э}{{\selectfont\char253}}1
	{ю}{{\selectfont\char254}}1
	{я}{{\selectfont\char255}}1
	{А}{{\selectfont\char192}}1
	{Б}{{\selectfont\char193}}1
	{В}{{\selectfont\char194}}1
	{Г}{{\selectfont\char195}}1
	{Д}{{\selectfont\char196}}1
	{Е}{{\selectfont\char197}}1
	{Ж}{{\selectfont\char198}}1
	{З}{{\selectfont\char199}}1
	{И}{{\selectfont\char200}}1
	{Й}{{\selectfont\char201}}1
	{К}{{\selectfont\char202}}1
	{Л}{{\selectfont\char203}}1
	{М}{{\selectfont\char204}}1
	{Н}{{\selectfont\char205}}1
	{О}{{\selectfont\char206}}1
	{П}{{\selectfont\char207}}1
	{Р}{{\selectfont\char208}}1
	{С}{{\selectfont\char209}}1
	{Т}{{\selectfont\char210}}1
	{У}{{\selectfont\char211}}1
	{Ф}{{\selectfont\char212}}1
	{Х}{{\selectfont\char213}}1
	{Ц}{{\selectfont\char214}}1
	{Ч}{{\selectfont\char215}}1
	{Ш}{{\selectfont\char216}}1
	{Щ}{{\selectfont\char217}}1
	{Ъ}{{\selectfont\char218}}1
	{Ы}{{\selectfont\char219}}1
	{Ь}{{\selectfont\char220}}1
	{Э}{{\selectfont\char221}}1
	{Ю}{{\selectfont\char222}}1
	{Я}{{\selectfont\char223}}1
}

\usepackage{graphicx}
\newcommand{\img}[3] {
	\begin{figure}[h!]
		\center{\includegraphics[height=#1]{img/#2}}
		\caption{#3}
		\label{img:#2}
	\end{figure}
}


\usepackage[justification=centering]{caption} % Настройка подписей float объектов

\usepackage[unicode,pdftex]{hyperref} % Ссылки в pdf
\hypersetup{hidelinks}

\usepackage{csvsimple}

\newcommand{\code}[1]{\texttt{#1}}

\usepackage{longtable}

\usepackage{array}
\usepackage{booktabs}
\usepackage{floatrow}

\floatsetup[longtable]{LTcapwidth=table}

% \def\UrlBreaks{\do\/\do-\do\_}

\makeatletter
\renewcommand*\l@chapter[2]{%
  \ifnum \c@tocdepth >\m@ne
    \addpenalty{-\@highpenalty}%
    \vskip 1.0em \@plus\p@
    \setlength\@tempdima{1.5em}%
    \begingroup
      \parindent \z@ \rightskip \@pnumwidth
      \parfillskip -\@pnumwidth
      \leavevmode \bfseries
      \advance\leftskip\@tempdima
      \hskip -\leftskip
      #1\nobreak\normalfont\leaders\hbox{$\m@th
        \mkern \@dotsep mu\hbox{.}\mkern \@dotsep
        mu$}\hfill\nobreak\hb@xt@\@pnumwidth{\hss #2}\par
      \penalty\@highpenalty
    \endgroup
  \fi}
\makeatother

\begin{document}



\begin{titlepage}
	\newgeometry{pdftex, left=2cm, right=2cm, top=2.5cm, bottom=2.5cm}
	\fontsize{12pt}{12pt}\selectfont
	\noindent \begin{minipage}{0.15\textwidth}
		\includegraphics[width=\linewidth]{img/b_logo.jpg}
	\end{minipage}
	\noindent\begin{minipage}{0.9\textwidth}\centering
		\textbf{Министерство науки и высшего образования Российской Федерации}\\
		\textbf{Федеральное государственное бюджетное образовательное учреждение высшего образования}\\
		\textbf{«Московский государственный технический университет имени Н. Э.~Баумана}\\
		\textbf{(национальный исследовательский университет)»}\\
		\textbf{(МГТУ им. Н. Э.~Баумана)}
	\end{minipage}
	
	\noindent\rule{18cm}{3pt}
	\newline\newline
	\noindent ФАКУЛЬТЕТ $\underline{\text{«Информатика и системы управления»~~~~~~~~~~~~~~~~~~~~~~~~~~~~~~~~~~~~~~~~~~~~~~~~~~~~~~~}}$ \newline\newline
	\noindent КАФЕДРА $\underline{\text{«Программное обеспечение ЭВМ и информационные технологии»~~~~~~~~~~~~~~~~~~~~~~~}}$\newline\newline\newline\newline\newline\newline\newline
	
	
	\begin{center}
		\noindent\begin{minipage}{1.3\textwidth}\centering
		\Large\textbf{   ~~~ Лабораторная работа №1}\newline
		\textbf{по курсу "Функциональное}\newline
		\textbf{и логическое программирование"}\newline\newline\newline
		\end{minipage}
	\end{center}
	
	\noindent\textbf{Тема} 			$\underline{\text{Списки в Lisp. Использование стандартных функций}}$\newline\newline
	\noindent\textbf{Студент} 		$\underline{\text{Ковалец К. Э.}}$\newline\newline
	\noindent\textbf{Группа} 		$\underline{\text{ИУ7-63Б}}$\newline\newline
	\noindent\textbf{Преподаватели} $\underline{\text{Толпинская Н. Б., Строганов Ю. В.}}$\newline
	
	\begin{center}
		\vfill
		Москва~---~\the\year
		~г.
	\end{center}
	\restoregeometry
\end{titlepage}



\setcounter{page}{2}
\chapter{Практические задания}

Практические задания приложены к отчету.

\chapter{Ответы на теоретические вопросы к лабораторной работе}

\section{Элементы языка: определение, синтаксис, представление в памяти}

\subsection{Определение}

Вся информация (данные и программы) в Lisp представляются в виде символьных выражений -- S-выражений. По определению:

\begin{center}
\captionsetup{justification=raggedright,singlelinecheck=off}
\begin{lstlisting}
	S-выражение ::= <атом> | <точечная пара>
\end{lstlisting}
\end{center}	

\textbf{Атомы} могут быть следующими.

\begin{enumerate}
	\item Символы (идентификаторы) -- синтаксически представляется как набор букв и цифр, начинающийся с буквы.
	\item Специальные символы -- T, Nil (используются для обозначения логи- ческих констант).
	\begin{itemize}
		\item T -- обозначает логическое значение <<Истина>>, истинным значением является все, отличное от Nil.
		\item Nil -- обозначает логическое значение <<Ложь>>, также обозначает пустой список.
	\end{itemize}
	\item Самоопределимые атомы -- натуральные числа, дробные числа, веще- ственные числа, строки -- последовательность символов, заключенных в двойные апострофы (например "abc").
\end{enumerate}

\textbf{Точечная пара} -- (A.B). Cтроится с помощью бинарного узла.

\begin{center}
\captionsetup{justification=raggedright,singlelinecheck=off}
\begin{lstlisting}
	Точечная пара ::= (<атом>.<атом>) | 
					  (<атом>.<точечная пара>) |
					  (<точечная пара>.<атом>) | 
					  (<точечная пара>.<точечная пара>)
\end{lstlisting}
\end{center}	

\textbf{Список} -- динамическая структура данных, которая может быть пустой или непустой. Если она не пустая, то состоит из двух элементов:

\begin{enumerate}
	\item голова -- любая структура;
	\item хвост -- список.
\end{enumerate}

\begin{center}
\captionsetup{justification=raggedright,singlelinecheck=off}
\begin{lstlisting}
	Список ::= <пустой список> | <непустой список>, где 
			   <пустой список> ::= () | Nil ,
			   <непустой список> ::= (<первый элемент>.<хвост>), 
			   <первый элемент> ::= <S-выражение>,
			   <хвост> ::= <список>.
\end{lstlisting}
\end{center}	

\subsection{Синтаксис}

Любая структура (точечная пара или список) заключается в круглые
скобки (A.B) -- точечная пара, (А) -- список из одного элемента, пустой список изображается как Nil или ().

Непустой список можно записать следующими образами: (А.(B.(C.(D())))) или (A B C D).

Элементы списка могут, в свою очередь, быть списками (любой список за- ключается в круглые скобки), например — (А (B C) (D (E))). Таким образом, синтаксически наличие скобок является признаком структуры — списка или точечной пары.

\subsection{Представление в памяти}

Любая непустая структура Lisp в памяти представляется списковой ячейкой, хранящей два указателя: на голову и хвост.

\clearpage

\begin{enumerate}
	\item (A.B) — точечная пара.
	
	\begin{figure}[h]
		\centering
		\includegraphics[scale=0.45]{img/picture1.png}
		\caption{Представление в памяти (A.B)}
		\label{fig:picture1}
	\end{figure} 

	\item (A B) — список из двух элементов.
	
	\begin{figure}[h]
		\centering
		\includegraphics[scale=0.5]{img/picture2.png}
		\caption{Представление в памяти (A B)}
		\label{fig:picture2}
	\end{figure} 
\end{enumerate}


\section{Особенности языка Lisp. Структура программы. Символ апостроф}

Особенности языка Lisp следующие.

\begin{enumerate}
	\item В Lisp используется символьная обработка.
	\item Программа может быть представлена в виде данных, поэтому она может изменять сама себя.
	\item Lisp является бестиповым языком, так как он работает только на указателях.
	\item Память выделяется блоками. LISP сам распределяет память.
	\item Программа и данные в LISP представлены списками.
\end{enumerate}

Символ апостроф (<<'>>) — блокирует вычисление своего аргумента. В качестве своего значения выдает сам аргумент, не вычисляя его. Перед константами -- числами и атомами T и Nil апостроф можно не ставить.

\section{Базис языка Lisp. Ядро языка}

\textbf{Базис языка} -- минимальный набор конструкций языка и структур данных, с помощью которых можно решить любую задачу.

Базис языка Lisp состоит из:

\begin{enumerate}
	\item структур, атомов;
	\item примитивных функций (car, cdr);
	\item специальных функций, управляющих обработкой структур, представляющих вычислимые выражения (quote).
\end{enumerate}

\textbf{Ядро} -- основные действия, которые наиболее часто используются. Ядро шире, чем базис.

\end{document}
