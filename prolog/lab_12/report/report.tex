\documentclass[a4paper,14pt, unknownkeysallowed]{extreport}

\usepackage{cmap} % Улучшенный поиск русских слов в полученном pdf-файле
\usepackage[T2A]{fontenc} % Поддержка русских букв
\usepackage[utf8]{inputenc} % Кодировка utf8
\usepackage[english,russian]{babel} % Языки: русский, английский
\usepackage{enumitem}


\usepackage{threeparttable}

\usepackage[14pt]{extsizes}

\usepackage{caption}
\captionsetup{labelsep=endash}
\captionsetup[figure]{name={Рисунок}}

% \usepackage{ctable}
% \captionsetup[table]{justification=raggedleft,singlelinecheck=off}

\usepackage{amsmath}

\usepackage{geometry}
\geometry{left=30mm}
\geometry{right=10mm}
\geometry{top=20mm}
\geometry{bottom=20mm}

\usepackage{titlesec}
\titleformat{\section}
	{\normalsize\bfseries}
	{\thesection}
	{1em}{}
\titlespacing*{\chapter}{0pt}{-30pt}{8pt}
\titlespacing*{\section}{\parindent}{*4}{*4}
\titlespacing*{\subsection}{\parindent}{*4}{*4}

\usepackage{setspace}
\onehalfspacing % Полуторный интервал

\frenchspacing
\usepackage{indentfirst} % Красная строка

\usepackage{titlesec}
\titleformat{\chapter}{\LARGE\bfseries}{\thechapter}{20pt}{\LARGE\bfseries}
\titleformat{\section}{\Large\bfseries}{\thesection}{20pt}{\Large\bfseries}

\usepackage{multirow}
\usepackage{listings}
\usepackage{xcolor}

% Для листинга кода:
\lstset{%
	language=Prolog,   					% выбор языка для подсветки	
	basicstyle=\small\sffamily,			% размер и начертание шрифта для подсветки кода
	numbers=left,						% где поставить нумерацию строк (слева\справа)
	numberstyle=\tiny,		     		% размер шрифта для номеров строк
	stepnumber=1,						% размер шага между двумя номерами строк
	numbersep=5pt,						% как далеко отстоят номера строк от подсвечиваемого кода
	frame=single,						% рисовать рамку вокруг кода
	tabsize=4,							% размер табуляции по умолчанию равен 4 пробелам
	captionpos=t,						% позиция заголовка вверху [t] или внизу [b]
	breaklines=true,					
	breakatwhitespace=true,				% переносить строки только если есть пробел
	backgroundcolor=\color{white},
	basicstyle=\footnotesize\ttfamily,
	keywordstyle=\color{blue},
	stringstyle=\color{red},
	commentstyle=\color{gray},
	showspaces=false,
    showstringspaces=false
}


\usepackage{pgfplots}
\usetikzlibrary{datavisualization}
\usetikzlibrary{datavisualization.formats.functions}


\lstset{
	literate=
	{а}{{\selectfont\char224}}1
	{б}{{\selectfont\char225}}1
	{в}{{\selectfont\char226}}1
	{г}{{\selectfont\char227}}1
	{д}{{\selectfont\char228}}1
	{е}{{\selectfont\char229}}1
	{ж}{{\selectfont\char230}}1
	{з}{{\selectfont\char231}}1
	{и}{{\selectfont\char232}}1
	{й}{{\selectfont\char233}}1
	{к}{{\selectfont\char234}}1
	{л}{{\selectfont\char235}}1
	{м}{{\selectfont\char236}}1
	{н}{{\selectfont\char237}}1
	{о}{{\selectfont\char238}}1
	{п}{{\selectfont\char239}}1
	{р}{{\selectfont\char240}}1
	{с}{{\selectfont\char241}}1
	{т}{{\selectfont\char242}}1
	{у}{{\selectfont\char243}}1
	{ф}{{\selectfont\char244}}1
	{х}{{\selectfont\char245}}1
	{ц}{{\selectfont\char246}}1
	{ч}{{\selectfont\char247}}1
	{ш}{{\selectfont\char248}}1
	{щ}{{\selectfont\char249}}1
	{ъ}{{\selectfont\char250}}1
	{ы}{{\selectfont\char251}}1
	{ь}{{\selectfont\char252}}1
	{э}{{\selectfont\char253}}1
	{ю}{{\selectfont\char254}}1
	{я}{{\selectfont\char255}}1
	{А}{{\selectfont\char192}}1
	{Б}{{\selectfont\char193}}1
	{В}{{\selectfont\char194}}1
	{Г}{{\selectfont\char195}}1
	{Д}{{\selectfont\char196}}1
	{Е}{{\selectfont\char197}}1
	{Ж}{{\selectfont\char198}}1
	{З}{{\selectfont\char199}}1
	{И}{{\selectfont\char200}}1
	{Й}{{\selectfont\char201}}1
	{К}{{\selectfont\char202}}1
	{Л}{{\selectfont\char203}}1
	{М}{{\selectfont\char204}}1
	{Н}{{\selectfont\char205}}1
	{О}{{\selectfont\char206}}1
	{П}{{\selectfont\char207}}1
	{Р}{{\selectfont\char208}}1
	{С}{{\selectfont\char209}}1
	{Т}{{\selectfont\char210}}1
	{У}{{\selectfont\char211}}1
	{Ф}{{\selectfont\char212}}1
	{Х}{{\selectfont\char213}}1
	{Ц}{{\selectfont\char214}}1
	{Ч}{{\selectfont\char215}}1
	{Ш}{{\selectfont\char216}}1
	{Щ}{{\selectfont\char217}}1
	{Ъ}{{\selectfont\char218}}1
	{Ы}{{\selectfont\char219}}1
	{Ь}{{\selectfont\char220}}1
	{Э}{{\selectfont\char221}}1
	{Ю}{{\selectfont\char222}}1
	{Я}{{\selectfont\char223}}1
}

\usepackage{graphicx}
\newcommand{\img}[3] {
	\begin{figure}[h!]
		\center{\includegraphics[height=#1]{img/#2}}
		\caption{#3}
		\label{img:#2}
	\end{figure}
}


\usepackage[justification=centering]{caption} % Настройка подписей float объектов

\usepackage[unicode,pdftex]{hyperref} % Ссылки в pdf
\hypersetup{hidelinks}

\usepackage{csvsimple}

\newcommand{\code}[1]{\texttt{#1}}

\usepackage{longtable}

\usepackage{array}
\usepackage{booktabs}
\usepackage{floatrow}

\floatsetup[longtable]{LTcapwidth=table}

% \def\UrlBreaks{\do\/\do-\do\_}

\makeatletter
\renewcommand*\l@chapter[2]{%
  \ifnum \c@tocdepth >\m@ne
    \addpenalty{-\@highpenalty}%
    \vskip 1.0em \@plus\p@
    \setlength\@tempdima{1.5em}%
    \begingroup
      \parindent \z@ \rightskip \@pnumwidth
      \parfillskip -\@pnumwidth
      \leavevmode \bfseries
      \advance\leftskip\@tempdima
      \hskip -\leftskip
      #1\nobreak\normalfont\leaders\hbox{$\m@th
        \mkern \@dotsep mu\hbox{.}\mkern \@dotsep
        mu$}\hfill\nobreak\hb@xt@\@pnumwidth{\hss #2}\par
      \penalty\@highpenalty
    \endgroup
  \fi}
\makeatother

\begin{document}



\begin{titlepage}
	\newgeometry{pdftex, left=2cm, right=2cm, top=2.5cm, bottom=2.5cm}
	\fontsize{12pt}{12pt}\selectfont
	\noindent \begin{minipage}{0.15\textwidth}
		\includegraphics[width=\linewidth]{img/b_logo.jpg}
	\end{minipage}
	\noindent\begin{minipage}{0.9\textwidth}\centering
		\textbf{Министерство науки и высшего образования Российской Федерации}\\
		\textbf{Федеральное государственное бюджетное образовательное учреждение высшего образования}\\
		\textbf{«Московский государственный технический университет имени Н. Э.~Баумана}\\
		\textbf{(национальный исследовательский университет)»}\\
		\textbf{(МГТУ им. Н. Э.~Баумана)}
	\end{minipage}
	
	\noindent\rule{18cm}{3pt}
	\newline\newline
	\noindent ФАКУЛЬТЕТ $\underline{\text{«Информатика и системы управления»~~~~~~~~~~~~~~~~~~~~~~~~~~~~~~~~~~~~~~~~~~~~~~~~~~~~~~~}}$ \newline\newline
	\noindent КАФЕДРА $\underline{\text{«Программное обеспечение ЭВМ и информационные технологии»~~~~~~~~~~~~~~~~~~~~~~~}}$\newline\newline\newline\newline\newline\newline\newline
	
	
	\begin{center}
		\noindent\begin{minipage}{1.3\textwidth}\centering
		\Large\textbf{   ~~~ Лабораторная работа №12}\newline
		\textbf{по курсу "Функциональное}\newline
		\textbf{и логическое программирование"}\newline\newline\newline
		\end{minipage}
	\end{center}
	
	\noindent\textbf{Тема} 			$\underline{\text{Работа программы на Prolog}}$\newline\newline
	\noindent\textbf{Студент} 		$\underline{\text{Ковалец К. Э.}}$\newline\newline
	\noindent\textbf{Группа} 		$\underline{\text{ИУ7-63Б}}$\newline\newline
	\noindent\textbf{Преподаватели} $\underline{\text{Толпинская Н. Б., Строганов Ю. В.}}$\newline
	
	\begin{center}
		\vfill
		Москва~---~\the\year
		~г.
	\end{center}
	\restoregeometry
\end{titlepage}



\setcounter{page}{2}

\chapter{Практические задания}

\section{Часть 1}

Составить программу, т.е. модель предметной области -- базу знаний, объединив в ней информацию-знания:

\begin{itemize}
	\item \textbf{«Телефонныйсправочник»}: Фамилия, №тел, Адрес–структура (Город, Улица, №дома, №квартиры);
	\item \textbf{«Автомобили»}: Фамилия\_владельца, Марка, Цвет, Стоимость, и др.;
	\item \textbf{«Вкладчики банков»}: Фамилия, Банк, счет, сумма, др.
\end{itemize}

Владелец может иметь несколько телефонов, автомобилей, вкладов (Факты). Используя правила, обеспечить возможность поиска:

\begin{enumerate}
	\item 
		\begin{enumerate}
			\item По № телефона найти: Фамилию, Марку автомобиля, Стоимость автомобиля (может быть несколько);
			\item Используя сформированное в предыдущем пункте правило, по № телефона найти: только Марку автомобиля (автомобилей может быть несколько).
		\end{enumerate}
	\item Используя простой, не составной вопрос: по Фамилии (уникальна в городе, но в разных городах есть однофамильцы) и Городу проживания найти: Улицу проживания, Банки, в которых есть вклады и № телефона.

\end{enumerate}


\begin{center}
\captionsetup{justification=raggedright,singlelinecheck=off}
\begin{lstlisting}[label=lst:parallel_processing,caption=Решение задания 1]
DOMAINS 
	surname = symbol.
	phone   = symbol.
	city    = symbol. 
	street  = symbol.
	house   = integer.
	flat    = integer.
	address_struct = address(city, street, house, flat).

	brand = symbol.
	color = symbol.
	price = integer.
	
	bank  = symbol.
	score = integer.
	sum   = integer.

PREDICATES
	phonebook(surname, phone, address_struct).
	car(surname, city, brand, color, price).
	bank_depositor(surname, city, bank, score, sum).
	
	func_1a(surname, phone, brand, price).
	func_1b(phone, brand).
	func_2(surname, symbol, phone, symbol, bank).

CLAUSES
	phonebook("Kishov",   "8-916-100-10-10", address("Moscow",     "Sunny Street",     1, 10)).
	phonebook("Kovalets", "8-916-200-20-20", address("Moscow",     "Happiness Street", 2, 20)).
	phonebook("Volkov",   "8-916-300-30-30", address("Moscow",     "Heroes Street",    3, 30)).
	phonebook("Volkov",   "8-111-111-11-11", address("Omsk",       "Traitors Street",  3, 30)).
	phonebook("Tsvetkov", "8-916-400-40-40", address("Petersburg", "Street of Love",   4, 40)).
	phonebook("Maslova",  "8-916-500-50-50", address("Moscow",     "Hope Street",      5, 50)).
	phonebook("Khamzina", "8-916-600-60-60", address("Orenburg",   "Lenin Street",     6, 60)).
	
	car("Volkov",   "Moscow",     "Mercedes", "white",  12000000).
	car("Kovalets", "Moscow",     "Porsche",  "black",  25000000).
	car("Kovalets", "Moscow",     "BMW",      "cherry", 800000).
	car("Tsvetkov", "Petersburg", "Lada",     "white",  450000).
	car("Maslova",  "Moscow",     "Mercedes", "white",  8000000).
	
	bank_depositor("Khamzina", "Orenburg", "Sberbank", 1001, 15000000).
	bank_depositor("Kovalets", "Moscow",   "Sberbank", 2002, 200000).
	bank_depositor("Volkov",   "Moscow",   "Sberbank", 4004, 25000).
	bank_depositor("Volkov",   "Omsk",     "Sberbank", 5005, 10).
	bank_depositor("Volkov",   "Moscow",   "VTB",      6006, 450000).
	bank_depositor("Maslova",  "Moscow",   "Tinkoff",  7007, 650000).
	
	
	func_1a(Surname, Phone, Brand, Price) :-  
		phonebook(Surname, Phone, _),
		car(Surname, _, Brand, _, Price). 
		
	func_1b(Phone, Brand) :- 
		func_1a(_, Phone, Brand, _).
	
	func_2(Surname, City, Phone, Street, Bank) :- 
		phonebook(Surname, Phone, address(City, Street, _, _)), 
		bank_depositor(Surname, City, Bank, _, _).
	
GOAL
	% func_1a(Surname, "8-916-200-20-20", Brand, Price).
	% func_1b("8-916-200-20-20", Brand).
	func_2("Volkov", "Moscow", Phone, Street, Bank).
\end{lstlisting}
\end{center}


\section{Часть 2}

Используя конъюнктивное правило и простой вопрос, обеспечить возможность поиска:

По Марке и Цвету автомобиля найти Фамилию, Город, Телефон и Банки, в которых владелец автомобиля имеет вклады. Лишней информации не находить и не передавать!!!

Владельцев может быть несколько (не более 3-х), один и ни одного.

\begin{itemize}
	\item Для каждого из трех вариантов словесно подробно описать порядок формирования ответа (в виде таблицы). При этом, указать -- отметить моменты очередного запуска алгоритма унификации и полный результат его работы. Обосновать следующий шаг работы системы. Выписать унификаторы -- подстановки. Указать моменты, причины и результат отката, если он есть.
	\item Для случая нескольких владельцев (2-х): приведите примеры (таблицы) работы системы при разных порядках следования в БЗ процедур, и знаний в них: («Телефонный справочник», «Автомобили», «Вкладчики банков», или: «Автомобили», «Вкладчики банков», «Телефонный справочник»). Сделайте вывод: Одинаковы ли: множество работ и объем работ в разных случаях?
	\item Оформите 2 таблицы, демонстрирующие порядок работы алгоритма унификации вопроса и подходящего заголовка правила (для двух случаев из пункта 2) и укажите результаты его работы: ответ и побочный эффект.
\end{itemize}

\begin{center}
\captionsetup{justification=raggedright,singlelinecheck=off}
\begin{lstlisting}[label=lst:parallel_processing,caption=Решение задания 2]
DOMAINS 
	surname = symbol.
	phone   = symbol.
	city    = symbol. 
	street  = symbol.
	house   = integer.
	flat    = integer.
	address_struct = address(city, street, house, flat).

	brand = symbol.
	color = symbol.
	price = integer.
	
	bank  = symbol.
	score = integer.
	sum   = integer.

PREDICATES
	phonebook(surname, phone, address_struct).
	car(surname, city, brand, color, price).
	bank_depositor(surname, city, bank, score, sum).
	
	func(brand, color, surname, city, phone, bank).

CLAUSES
	phonebook("Kishov",   "8-916-100-10-10", address("Moscow",     "Sunny Street",     1, 10)).
	phonebook("Kovalets", "8-916-200-20-20", address("Moscow",     "Happiness Street", 2, 20)).
	phonebook("Volkov",   "8-916-300-30-30", address("Moscow",     "Heroes Street",    3, 30)).
	phonebook("Volkov",   "8-111-111-11-11", address("Omsk",       "Traitors Street",  3, 30)).
	phonebook("Tsvetkov", "8-916-400-40-40", address("Petersburg", "Street of Love",   4, 40)).
	phonebook("Maslova",  "8-916-500-50-50", address("Moscow",     "Hope Street",      5, 50)).
	phonebook("Khamzina", "8-916-600-60-60", address("Orenburg",   "Lenin Street",     6, 60)).
	
	car("Volkov",   "Moscow",     "Mercedes", "white",  12000000).
	car("Kovalets", "Moscow",     "Porsche",  "black",  25000000).
	car("Kovalets", "Moscow",     "BMW",      "cherry", 800000).
	car("Tsvetkov", "Petersburg", "Lada",     "white",  450000).
	car("Maslova",  "Moscow",     "Mercedes", "white",  8000000).
	
	bank_depositor("Khamzina", "Orenburg", "Sberbank", 1001, 15000000).
	bank_depositor("Kovalets", "Moscow",   "Sberbank", 2002, 200000).
	bank_depositor("Volkov",   "Moscow",   "Sberbank", 4004, 25000).
	bank_depositor("Volkov",   "Omsk",     "Sberbank", 5005, 10).
	bank_depositor("Volkov",   "Moscow",   "VTB",      6006, 450000).
	bank_depositor("Maslova",  "Moscow",   "Tinkoff",  7007, 650000).
	
	func(Brand, Color, Surname, City, Phone, Bank) :- 
		car(Surname, City, Brand, Color, _), 
		phonebook(Surname, Phone, address(City, _, _, _)),
		bank_depositor(Surname, City, Bank, _, _).

GOAL
	% Нет владельцев
	% func("Porsche", "white", Surname, City, Phone, Bank).
	% Один владелец
	% func("BMW", "cherry", Surname, City, Phone, Bank).
	% Два владельца
	func("Mercedes", "white", Surname, City, Phone, Bank).
\end{lstlisting}
\end{center}

\begin{figure}[h]
	\centering
	\includegraphics[scale=0.5]{img/result1.png}
	\caption{Результат работы программы (1 владелец)}
	\label{fig:result1}
\end{figure}

\begin{figure}[h]
	\centering
	\includegraphics[scale=0.5]{img/result2.png}
	\caption{Результат работы программы (2 владельца)}
	\label{fig:result2}
\end{figure}

\end{document}
